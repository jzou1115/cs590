\documentclass[11pt]{article}

\usepackage{amsmath, amssymb}

\title{\textbf{Graph-Based Representations of the Protein Design Problem}}
\author{Adi Mukund \and Jennifer Zou}
\date{}

\begin{document}
	
	\maketitle	
	
	\section{Context}
	
	The protein design problem can be described as follows. Given a set of 
	backbone coordinates $\mathbf{c} = (\vec{c_1}, \vec{c_2 }, \dots, \vec{c_n})$
	and rotamer library $R$ of length $r$, identify the optimal rotamer
	assignment sequence $\vec{r} = (r_1, r_2,\dots, r_n)$, $r_i \in R$, 
	$1 \leq i \leq n$ according to an energy function $E(\vec{r})$. This assignment
	sequence is known as the global minimum energy conformation (GMEC). 
	This problem has been proven to be NP-hard \cite{PW02}, and algorithms such 
	as DEE \cite{BD97} have been created to allow for combinatorial pruning of 
	the residue search space and make the design problem computationally tractable.
	However, such algorithms cannot guarantee a time complexity less than the 
	worst-case $O(nr^n)$. 
	
	Recent graph based algorithms such as BWM* \cite{DJJG15} use sparse residue 
	interaction graphs in order to more efficiently compute functions over the residue
	space and identify optimal assignments more rapidly. Such graph-based algorithms
	have been able to achieve combinatorial speedups while maintaining provable
	accuracy and returning ensemble of minimum energy conformations. 
	
	Probabilistic models of protein design assign a probability distribution 
	for rotamers in each position of the protein sequence.  In algorithms such as
	belief propagation, the beliefs or the approximate marginal probabilities
	of each rotamer are computed iteratively.  These approximations are computationally
	less intense, but they are not always provably accurate \cite{KLX08}. Furthermore,
	such algorithms are only guaranteed to converge on tree graphs, which are not
	commonly observed in natural proteins \cite{FLY09}. 
	
	\section{Goals}
	
	\subsection{Provable Convergence}
	
	The project will attempt to develop provably accurate algorithms utilizing
	belief propagation and graph cuts on sparse residue interaction graphs to solve
	the protein design problem. The major obstacle will be maintaining provable
	convergence in the presence of loops in the sparse graph. We will attempt to do
	this by:
	
	\begin{enumerate}
		\item Using graph cuts and dynamic programming in order to treat loops within
		the residue graphs as subproblems and maintain provable accuracy. 
		
		\item Continually pruning edges until the graph is sufficiently sparse, 
		maintaining an error bound in order to see the maximum difference $\epsilon$
		in energy between the full GMEC and sparse GMEC, and enumerating all low-
		energy conformations produced by the sparse interaction graph within $\epsilon$
		of the sparse GMEC.  
		
		\item Attempting to find generalized belief propagation algorithms that provably
		converge on graphs that contain loops, and applying those algorithms to the
		protein design problem.
		
		\item Investigating other properties of the sparse residue interaction graph
		such as planarity and seeing if these allow for provably accurate algorithms.
	\end{enumerate}
	
	\subsection{Machine Learning Models of Protein Stability}
	We hope to train energy functions in order to
	improve the speed of residue-assignment algorithms. Current energy
	functions include pairwise terms and are frequently slow enough that running
	them on a large number of possible conformations is impossible; machine 
	learning models could yield significant speed improvements. 
	
	Machine learning methods may also increase accuracy of residue-assignment 
	algorithms.  When creating sparse residue interaction graphs, assumptions
	must be made about which interactions to include.  These assumptions are
	usually made with distance or energy cutoffs, which may be biased. Furthermore,
	many energy functions do not include terms for entropy, which can decrease the
	accuracy of residue assignment algorithms \cite{SKNCAKRST13}. By training
	directly with free energy data, these issues may be addressed.
	
	Using protein sequence and thermodynamic data from the Protein Data Bank and
	ProTherm datavases, we hope to train energy functions to predict energies of
	novel sequences. A model such as a neural network may be ideal due to its 
	flexibility and ability to capture nonlinear relationships.
	
	
	\subsection{Test Accuracy}
	
	Once we have implemented our approximation algorithms, we would like to test
	them for accuracy and compare the results to known algorithms.  We would also
	like to test the effectiveness of the algorithms different protein families
	and for different regions (surface, boundary, and core) of proteins.  This
	will allow us to assess whether certain algorithms are better for specific
	types of structures.
	
	\begin{thebibliography}{99}
		\bibitem{BD97} Dahiyat, B. I. De Novo Protein Design: Fully Automated Sequence
		Selection. Science 278, 82-87 (1997)
		
		\bibitem{DJJG15} Jou, J.D. Jain, S. Georgiev, I. Donald, B.R. BWM*: A Novel,
		Provable, Ensemble-based Dynamic Programming Algorithm for Sparse 
		Approximations of Computational Protein Design. RECOMB (2015) Warsaw, Poland.
		April 12, 2015 (In Press)
		
		\bibitem{KLX08} Kamisetty H1, Xing EP, Langmead CJ. Free energy estimates of
		all-atom protein structures using generalized belief propagation. J Comput
		Biol. 2008 Sep;15(7):755-66. doi: 10.1089/cmb.2007.0131.
		
		\bibitem{FLY09} M. Fromer, C. Yanover, and M. Linial. Design of multispecific
		protein sequences using probabilistic graphical modeling. Proteins 75;3(2009
		May 15):682-705.
		
		\bibitem{PW02} Pierce, Niles A. and Winfree, Erik. Protein Design is NP-hard 
		Protein Eng. (2002) 15 (10): 779-782 doi:10.1093/protein/15.10.779
		
		\bibitem{SKNCAKRST13} Silver, N.W. King, B.M. Nalam, M.N. Cao, H. Ali, 
		A. Kiran Kumar Reddy, G.S. Rana, T.M. Schiffer, C.A. Tidor, B. Efficient
		Computation of Small-Molecule Configurational Binding Entropy and Free
		Energy Changes by Ensemble Enumeration. J Chem Theory Comput 9, 5098-
		5115 (2013).
		
	\end{thebibliography}
	
	
\end{document}