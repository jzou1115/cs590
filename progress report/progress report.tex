\documentclass[11pt]{article}

\usepackage{amsmath, amssymb}

\title{\textbf{Progress Report: Graph-Based Representations of the Protein 
		Design Problem}}
\author{Adi Mukund \and Jennifer Zou}
\date{}

\begin{document}
	
	\maketitle

	This project consists of two main components: (1) an attempt to apply graph
	cuts to the residue interaction graph in order to see if such algorithms 
	hold any potential for developing new, efficient approaches to solving the
	protein design problem, and (2) a graph decomposition-based modification of
	cluster expansion in order to retain the computational benefits that cluster
	expansion provides while minimizing the gap between the values returned by
	cluster expansion and more traditional energy functions. 

	
	\section{Graph Cuts and the GMEC}
	
	The main idea behind this component of the project is to try to apply graph
	cut-based algorithms to the protein design problem. We began by attempting
	to characterize the protein design problem as a graph cut problem and 
	identifying relevant algorithms that might allow efficient approximations
	of the GMEC. 
	
	The protein design problem is most accurately represented not as solely a
	graph cut problem, but as a graph labeling problem, where each rotamer is a label.
	The Graph Labeling (GL) problem can be stated as follows: classify a set 
	$\mathcal{V}$ of $n$ objects by assigning to each object a label from a given
	set $\mathcal{L}$ of labels, given a weighted graph
	$\mathcal{G}=(\mathcal{V},\mathcal{E},w)$. For each $p \in \mathcal{V}$ there
	is a label cost $\textbf{c}_{p}(a) \geq 0$ for assigning the label $a=f_p$ to
	$p$, and for every edge $pq$ there is a pairwise cost
	$\textbf{c}_{pq}(a,b) =  w_{pq}d_{pq}(ab)$ where $d_{pq}(ab)$ is the distance
	between (or cost of) label $a$ on vertex $p$ and $b$ on vertex $q$. Thus,
	the cost of a labeling $f$ is as follows:
		
	\begin{equation} \label{k_cost}
			\text{COST}(f) = 
			\sum\limits_{p \in \mathcal{V}} \textbf{c}_{p}(f_p) + 
			\sum\limits_{(p,q) \in \mathcal{E}} w_{pq}d_{pq}(f_p,f_q)
		\end{equation}
		
	
	The most promising algorithm for an efficient solution was \cite{Karmarkar}, 
	which provides an approximation algorithm based on graph cuts for the 
	nonmetric labeling problem, which requires a distance function $d(a,b)$ such
	that $d(a,b)=0 \iff a=b$ and $d(a,b) \geq 0$. The algorithm provides a labeling 
	with a cost that is an $f$-approximation to the minimum-cost labeling, where
	$f = \frac{d_{max}}{d_{min}}$, where $d_{max}$ is the maximum distance between
	any two rotamers and $d_{min}$ is the minimum distance. 
	
	The steps over the next few weeks are as follows:
	\begin{enumerate}
		\item Implement the algorithm as an extension to OSPREY

		\item Identify how the conformations from by the algorithm differ from 
		those provided by OSPREY. 
		
		\item Determine if the $f$-approximation is sufficiently tight so as to
		produce useful results. 
	\end{enumerate} 
	
	\section{Cluster Expansion}
	
	
	\begin{thebibliography}{99}
		\bibitem{Karmarkar} Komodakis, N.; Tziritas, G., "Approximate Labeling
		via Graph Cuts Based on Linear Programming," Pattern Analysis and Machine Intelligence, IEEE Transactions on , vol.29, no.8, pp.1436,1453, Aug. 2007
		doi: 10.1109/TPAMI.2007.1061		
	\end{thebibliography}

\end{document}