\documentclass[11pt]{article}

\usepackage{amsmath, amssymb, booktabs}

\title{\textbf{Novel Approaches to the Protein Design Problem}}
\author{Adi Mukund \and Jennifer Zou}
\date{April 20, 2015}

\begin{document}
	
	\maketitle
	
	\section{Introduction}
	The protein design problem can be described as follows. Given a set of 
	backbone coordinates $\mathbf{c} = (\vec{c_1}, \vec{c_2 }, \dots, \vec{c_n})$
	and rotamer library $R$ of length $r$, identify the optimal rotamer
	assignment sequence $\vec{r} = (r_1, r_2,\dots, r_n)$, $r_i \in R$, 
	$1 \leq i \leq n$ according to an energy function $E(\vec{r})$. This assignment
	sequence is known as the global minimum energy conformation (GMEC). 
	This problem has been proven to be NP-hard \cite{PW02}, and algorithms such 
	as DEE \cite{BD97} have been created to allow for combinatorial pruning of 
	the residue search space and make the design problem computationally tractable.
	However, such algorithms cannot guarantee a time complexity less than the 
	worst-case $O(nr^n)$. 
	
	Recent graph based algorithms such as BWM* \cite{DJJG15} use sparse residue 
	interaction graphs in order to more efficiently compute functions over the residue
	space and identify optimal assignments more rapidly. Such graph-based algorithms
	have been able to achieve combinatorial speedups while maintaining provable
	accuracy and returning ensemble of minimum energy conformations. 	
	
	Probabilistic models of protein design assign a probability distribution 
	for rotamers in each position of the protein sequence.  In algorithms such as
	belief propagation, the beliefs or the approximate marginal probabilities
	of each rotamer are computed iteratively.  These approximations are computationally
	less intense, but they are not always provably accurate \cite{KLX08}. Furthermore,
	such algorithms are only guaranteed to converge on tree graphs, which are not
	commonly observed in natural proteins \cite{FLY09}. 

	\section{Graph Cuts and the GMEC}
	
	\subsection{The Graph Labeling Problem}
	
	The goal of this project was to apply graph-based algorithms to the protein
	design problem. We began by attempting to characterize the protein design problem
	as a graph cut problem and identifying relevant algorithms that might allow
	efficient approximations of the GMEC. 
	
	The protein design problem is most accurately represented not as solely a
	graph cut problem, but as a graph labeling problem, where each rotamer is a label.
	The Graph Labeling (GL) problem can be stated as follows: classify a set 
	$\mathcal{V}$ of $n$ objects by assigning to each object a label from a given
	set $\mathcal{L}$ of labels, given a weighted graph
	$\mathcal{G}=(\mathcal{V},\mathcal{E},w)$. For each $p \in \mathcal{V}$ there
	is a label cost $\textbf{c}_{p}(a) \geq 0$ for assigning the label $a=f_p$ to
	$p$, and for every edge $pq$ there is a pairwise cost
	$\textbf{c}_{pq}(a,b) =  w_{pq}d_{pq}(ab)$ where $d_{pq}(ab)$ is the distance
	between (or cost of) label $a$ on vertex $p$ and $b$ on vertex $q$. Thus,
	the cost of a labeling $f$ is as follows:
		
	\begin{equation} \label{k_cost}
			\text{COST}(f) = 
			\sum\limits_{p \in \mathcal{V}} \textbf{c}_{p}(f_p) + 
			\sum\limits_{(p,q) \in \mathcal{E}} w_{pq}d_{pq}(f_p,f_q)
		\end{equation}

	The most promising algorithm for an efficient solution was \cite{Karmarkar}, 
	which provides an approximation algorithm based on graph cuts for the 
	nonmetric labeling problem, which requires a distance function $d(a,b)$ such
	that $d(a,b)=0 \iff a=b$ and $d(a,b) \geq 0$. The algorithm provides a labeling 
	with a cost that is an $f$-approximation to the minimum-cost labeling, where
	$f = \frac{d_{max}}{d_{min}}$, where $d_{max}$ is the maximum distance between
	any two rotamers and $d_{min}$ is the minimum distance. 

	\subsection{Application to the Protein Design Problem}
	
	The internal energy of a protein can easily be modeled by equation \ref{k_cost},
	where the function $\textbf{c}_{p}(f_p)$ is held to represent the internal
	energy of a rotamer $f_p$ at position $p$ and the function 
	$w_{pq}d_{pq}(f_p, f_q)$ is held to represent the pairwise interaction of a
	rotamer $f_p$ at position $p$ and a rotamer $f_q$ at position $q$. 	In addition,
	each residue is modeled by a single node in the vertex set $\mathcal{V}$, and
	interactions between residues are represented by edges between nodes in the edge
	set $\mathcal{E}$. 
	
	In order to maintain the distance constraint 
	$\textbf{c}_{pq}(a,b) =  w_{pq}d_{pq}(ab)$, the label set $L$ was set as the
	Cartesian product $R \times \mathcal{V}$ of the set of all rotamers with the set
	of all positions. Thus, a given label $l \in \mathcal{L}$ represents a specific
	rotamer at a particular position. The pairwise interaction between a label and
	itself can then be set as zero. In order to ensure that a rotamer for position 1
	was not assigned to position 2, the cost of labeling a rotamer to the wrong
	position was set to be prohibitively high. 
	
	\section{Methods}
	\subsection{Selection of proteins}
	\subsection{Generation of pairwise interaction matrices}
	OSPREY (\cite{OS1}, \cite{OS2}) was used to generate pairwise interaction matrices
	and provide a baseline against which to compare the results generated by the graph
	-based algorithm.
	
	\subsection{Optimizing the metric labeling problem}
	The FastPD Markov Random Field optimization library 
	(\cite{Karmarkar} and \cite{Komodakis}) was used in order to test the graph
	algorithm. The FastPD algorithm transforms the initial GL problem into the
	minimization of a linear program according to \cite{CKNZ}. 
	
	\begin{equation}
		\begin{split}
		min & \sum_{p \in V} \sum_{a\in L} c_p(a)x_p(a)+ \sum_{(p,q) \in E} w_{pq}
		 \sum_{a,b \in L}d(a,b) x_{pq}(a,b)\\
		s.t. & \sum_a x_p(a) =1 \qquad \forall p \in V\\
		& \sum_a x_{pq}(a,b)=x_q(b) \qquad \forall b \in L, (p,q) \in E\\
		& \sum_b x_{pq}(a,b)=x_p(a) \qquad \forall a \in L, (p,q) \in E
		\end{split}
	\label{linear}
	\end{equation}
	
	Equation \ref{linear} shows a formulation of an integer linear program, where
	$x_p(a)$ is a binary variable that indicates whether vertex p is assigned to
	label a, and $x_{pq}(a,b)$ indicates that vertices p and q are assigned the
	labels a and b.  The first constraint requires each vertex to be assigned to
	one label, and the other constraints maintains consistency between the labeled
	vertices and edges.
	
	Optimizing the cost of the metric labeling problem is NP-hard, and many
	approximation algorithms ($\alpha$-expansion, $\alpha$-$\beta$-swap) either
	assume metric distances that satisfy the triangle inequality or provide no
	guarantees about optimality.  Since pairwise interaction energies between
	protein residues are inherently nonmetric distances, these algorithms are not
	adequate for the protein design problem.  
	
	FastPD only requires a nonmetric distance function that satisfies $d(a,b)$ such
	that $d(a,b)=0 \iff a=b$ and $d(a,b) \geq 0$.  Since the pairwise interaction
	energy between a residue and itself is zero, and all energies are greater than
	zero, this algorithm can be applied to the protein design problem.  
	
	Relaxation of the integer linear program constraints transforms the NP-hard
	optimization problem into one that can be solvable in polynomial time.  The 
	FastPD algorithm modifies the constraints to 
	$x_p(\cdot) \ge 0$ and $x_{pq}(\cdot, \cdot) \ge 0$.  

	\subsection{The Primal-Dual Schema}	
	
	The FastPD algorithm utilizes the primal-dual schema in order to rapidly obtain
	an $f$-approximation to the optimal solution. Given a primal program
	\begin{gather*}
	\text{min }\textbf{c}^T\textbf{x} \\
	\text{s.t. }\textbf{Ax} = \textbf{b}, \textbf{x} \geq 0
	\end{gather*}
	the dual program can be formulated as follows:
	\begin{gather*}
	\text{max } \textbf{b}^T\textbf{y} \\
	\text{s.t. }\textbf{A}^T\textbf{y} \leq \textbf{c} 
	\end{gather*}		

	The primal-dual principle states that if $\textbf{x}$ and $\textbf{y}$ are
	solutions satisfying $\textbf{c}^T\textbf{x} \leq f \cdot \textbf{b}^T\textbf{y}$,
	then $\textbf{x}$ is an $f$-approximation to the optimal solution $\textbf{x}*$.
	FastPD starts with initial guesses for $\textbf{x}$ and $\textbf{y}$ and 
	repeatedly improves them by making every variable in the program a node in the
	graph and using a max-flow/min-cut algorithm to alter the assigned values to
	each variable. The algorithm converges to an approximation factor
	$f_{app} = 2 \frac{d_{max}}{d_{min}}$ where $d_{max}$ is the largest distance
	between two labels and $d_{min}$ is the smallest distance between two labels. 
	
	Thus, the FastPD algorithm returns a conformation whose energy is an $f$-
	approximation to the GMEC. This project aimed to test whether or not this would
	be sufficiently accurate so as to produce useful results. 

\section{Results}
After 100 iterations, FastPD assigned residues to selected positions.  In Tables \ref{dhfr}, \ref{1cc8}, and \ref{1fsv}, the results for these assignments are summarized.  The first column in each table contains the position to be optimized, and the second contains the assigned label.  The assigned label format indicates the amino acid and position assigned to the site ([amino acid]-[position]).  Many of the selected labels did not correspond to the correct position or interest, indicating limitations of the algorithm. 
\begin{table}[h]
\center
\begin{tabular}{@{}ll@{}}
\toprule
DHFR     &                \\ \midrule
\textbf{Position} & \textbf{Assigned Label} \\
5        & Leu-0          \\ \bottomrule
\end{tabular}
\caption{FastPD assignments for DHFR}
\label{dhfr}
\end{table}

%%%%%%%%%%%%%%%%%%%%%%%%%%%%%%
\begin{table}[h]
\center
\begin{tabular}{@{}ll@{}}
\toprule
1CC8     &                \\ \midrule
\textbf{Position} & \textbf{Assigned Label} \\
0        & Ser-2          \\
1        & Val-6          \\
2        & Val-6          \\
3        & Ala-6          \\
4        & Ley-6          \\
5        & Ala-6          \\
6        & Leu-6          \\ \bottomrule
\end{tabular}
\caption{FastPD assignments for 1CC8}
\label{1cc8}
\end{table}
%%%%%%%%%%%%%%%%%

\begin{table}[h]
\center
\begin{tabular}{@{}ll@{}}
\toprule
1FSV     &                \\ \midrule
\textbf{Position} & \textbf{Assigned Label} \\
10       & glu-27         \\
27       & his-27         \\ \bottomrule
\end{tabular}
\caption{FastPD assignments for 1FSV}
\label{1fsv}
\end{table}


	\section{Limitations}
	\subsection{F approximation error}
	\subsection{Multilabel assignment}
	\subsection{Density of optimal solutions}
	\section{Conclusion}
	
	\begin{thebibliography}{99}
	
	\bibitem {PW02} Pierce, Niles A. and Winfree, Erik. Protein Design is NP-hard 
	Protein Eng. (2002) 15 (10): 779-782 doi:10.1093/protein/15.10.779
	
	\bibitem{BD97} Dahiyat, B. I. De Novo Protein Design: Fully Automated Sequence
	Selection. Science 278, 82-87 (1997)	

	\bibitem{DJJG15} Jou, J.D. Jain, S. Georgiev, I. Donald, B.R. BWM*: A Novel,
	Provable, Ensemble-based Dynamic Programming Algorithm for Sparse 
	Approximations of Computational Protein Design. RECOMB (2015) Warsaw, Poland.
	April 12, 2015 (In Press)

	\bibitem{KLX08} Kamisetty H1, Xing EP, Langmead CJ. Free energy estimates of
	all-atom protein structures using generalized belief propagation. J Comput
	Biol. 2008 Sep;15(7):755-66. doi: 10.1089/cmb.2007.0131.
		
	\bibitem{FLY09} M. Fromer, C. Yanover, and M. Linial. Design of multispecific
	protein sequences using probabilistic graphical modeling. Proteins 75;3(2009
	May 15):682-705.
	
	\bibitem{OS1} C. Chen, I. Georgiev, A. C. Anderson, and B. R. Donald. 
	Computational structure-based redesign of enzyme activity. PNAS USA, 106(10):
	3764–3769, 2009.

	\bibitem{OS2} P. Gainza, K. E. Roberts, I. Georgiev, R. H. Lilien, D. A. Keedy,
	C. Chen, F. Reza, A. C. Anderson, D. C. Richardson, J. S. Richardson, and B. R.
	Donald. OSPREY: Protein design with ensembles, flexibility, and provable
	algorithms. Methods in Enzymology, 523:87-107, 2013.
	
	\bibitem{Karmarkar} Komodakis, N.; Tziritas, G., "Approximate Labeling
	via Graph Cuts Based on Linear Programming," Pattern Analysis and Machine
	Intelligence, IEEE Transactions on , vol.29, no.8, pp.1436,1453, Aug. 2007
	doi: 10.1109/TPAMI.2007.1061	
	
	\bibitem{Komodakis} N. Komodakis, G. Tziritas and N. Paragios, "Performance vs
	Computational Efficiency for Optimizing Single and Dynamic MRFs: Setting the 
	State of the Art with Primal Dual Strategies". Computer Vision and Image
	Understanding, 2008 (Special Issue on Discrete Optimization in Computer Vision).
	
	\bibitem{CKNZ} C. Chekuri, S. Khanna, J. Naor, and L. Zosin, “Approximation
	Algorithms for the Metric Labeling Problem via a New Linear Programming
	Formulation,” Proc. 12th Ann. ACM-SIAM Symp. Discrete Algorithms, pp. 109-118,
	2001.
	
	\end{thebibliography}

\end{document}

