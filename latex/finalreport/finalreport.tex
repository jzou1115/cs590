\documentclass[11pt]{article}

\usepackage{amsmath, amssymb}

\title{\textbf{Novel Approaches to the Protein Design Problem}}
\author{Adi Mukund \and Jennifer Zou}
\date{April 20, 2015}

\begin{document}
	
	\maketitle
	
	\section{Introduction}
	The protein design problem can be described as follows. Given a set of 
	backbone coordinates $\mathbf{c} = (\vec{c_1}, \vec{c_2 }, \dots, \vec{c_n})$
	and rotamer library $R$ of length $r$, identify the optimal rotamer
	assignment sequence $\vec{r} = (r_1, r_2,\dots, r_n)$, $r_i \in R$, 
	$1 \leq i \leq n$ according to an energy function $E(\vec{r})$. This assignment
	sequence is known as the global minimum energy conformation (GMEC). 
	This problem has been proven to be NP-hard \cite{PW02}, and algorithms such 
	as DEE \cite{BD97} have been created to allow for combinatorial pruning of 
	the residue search space and make the design problem computationally tractable.
	However, such algorithms cannot guarantee a time complexity less than the 
	worst-case $O(nr^n)$. 
	
	Recent graph based algorithms such as BWM* \cite{DJJG15} use sparse residue 
	interaction graphs in order to more efficiently compute functions over the residue
	space and identify optimal assignments more rapidly. Such graph-based algorithms
	have been able to achieve combinatorial speedups while maintaining provable
	accuracy and returning ensemble of minimum energy conformations. 	
	
	Probabilistic models of protein design assign a probability distribution 
	for rotamers in each position of the protein sequence.  In algorithms such as
	belief propagation, the beliefs or the approximate marginal probabilities
	of each rotamer are computed iteratively.  These approximations are computationally
	less intense, but they are not always provably accurate \cite{KLX08}. Furthermore,
	such algorithms are only guaranteed to converge on tree graphs, which are not
	commonly observed in natural proteins \cite{FLY09}. 

	\section{Graph Cuts and the GMEC}
	
	\subsection{The Graph Labeling Problem}
	
	The goal of this project was to apply graph-based algorithms to the protein
	design problem. We began by attempting to characterize the protein design problem
	as a graph cut problem and identifying relevant algorithms that might allow
	efficient approximations of the GMEC. 
	
	The protein design problem is most accurately represented not as solely a
	graph cut problem, but as a graph labeling problem, where each rotamer is a label.
	The Graph Labeling (GL) problem can be stated as follows: classify a set 
	$\mathcal{V}$ of $n$ objects by assigning to each object a label from a given
	set $\mathcal{L}$ of labels, given a weighted graph
	$\mathcal{G}=(\mathcal{V},\mathcal{E},w)$. For each $p \in \mathcal{V}$ there
	is a label cost $\textbf{c}_{p}(a) \geq 0$ for assigning the label $a=f_p$ to
	$p$, and for every edge $pq$ there is a pairwise cost
	$\textbf{c}_{pq}(a,b) =  w_{pq}d_{pq}(ab)$ where $d_{pq}(ab)$ is the distance
	between (or cost of) label $a$ on vertex $p$ and $b$ on vertex $q$. Thus,
	the cost of a labeling $f$ is as follows:
		
	\begin{equation} \label{k_cost}
			\text{COST}(f) = 
			\sum\limits_{p \in \mathcal{V}} \textbf{c}_{p}(f_p) + 
			\sum\limits_{(p,q) \in \mathcal{E}} w_{pq}d_{pq}(f_p,f_q)
		\end{equation}

	The most promising algorithm for an efficient solution was \cite{Karmarkar}, 
	which provides an approximation algorithm based on graph cuts for the 
	nonmetric labeling problem, which requires a distance function $d(a,b)$ such
	that $d(a,b)=0 \iff a=b$ and $d(a,b) \geq 0$. The algorithm provides a labeling 
	with a cost that is an $f$-approximation to the minimum-cost labeling, where
	$f = \frac{d_{max}}{d_{min}}$, where $d_{max}$ is the maximum distance between
	any two rotamers and $d_{min}$ is the minimum distance. 

	\subsection{Application to the Protein Design Problem}
	
	The internal energy of a protein can easily be modeled by equation \ref{k_cost},
	where the function $\textbf{c}_{p}(f_p)$ is held to represent the internal
	energy of a rotamer $f_p$ at position $p$ and the function 
	$w_{pq}d_{pq}(f_p, f_q)$ is held to represent the pairwise interaction of a
	rotamer $f_p$ at position $p$ and a rotamer $f_q$ at position $q$. 	In addition,
	each residue is modeled by a single node in the vertex set $\mathcal{V}$, and
	interactions between residues are represented by edges between nodes in the edge
	set $\mathcal{E}$. 
	
	In order to maintain the distance constraint 
	$\textbf{c}_{pq}(a,b) =  w_{pq}d_{pq}(ab)$, the label set $L$ was set as the
	Cartesian product $R \times \mathcal{V}$ of the set of all rotamers with the set
	of all positions. Thus, a given label $l \in \mathcal{L}$ represents a specific
	rotamer at a particular position. The pairwise interaction between a label and
	itself can then be set as zero. In order to ensure that a rotamer for position 1
	was not assigned to position 2, the cost of labeling a rotamer to the wrong
	position was set to be prohibitively high. 
	
	\section{Methods}
	
	OSPREY (\cite{OS1}, \cite{OS2}) was used to generate pairwise interaction matrices
	and provide a baseline against which to compare the results generated by the graph
	-based algorithm. The FastPD Markov Random Field optimization library 
	(\cite{Karmarkar} and \cite{Komodakis}) was used in order to test the graph
	algorithm. 
	
	The FastPD algorithm transforms the initial GL problem into the minimization
	of a linear program according to \cite{CKNZ}. The primal-dual schema is
	subsequently used in order to rapidly obtain an $f$-approximation to the 
	optimal solution, with graph-cuts being used to repeatedly refine approximations
	to the primal and dual solutions to the linear program minimization problem. 
	
	\section{Results}
	
	\begin{enumerate}
		\item DHFR
		\item 1CC8
		\item 1FSV
	\end{enumerate}

	\begin{thebibliography}{99}
	
	\bibitem {PW02} Pierce, Niles A. and Winfree, Erik. Protein Design is NP-hard 
	Protein Eng. (2002) 15 (10): 779-782 doi:10.1093/protein/15.10.779
	
	\bibitem{BD97} Dahiyat, B. I. De Novo Protein Design: Fully Automated Sequence
	Selection. Science 278, 82-87 (1997)	

	\bibitem{DJJG15} Jou, J.D. Jain, S. Georgiev, I. Donald, B.R. BWM*: A Novel,
	Provable, Ensemble-based Dynamic Programming Algorithm for Sparse 
	Approximations of Computational Protein Design. RECOMB (2015) Warsaw, Poland.
	April 12, 2015 (In Press)

	\bibitem{KLX08} Kamisetty H1, Xing EP, Langmead CJ. Free energy estimates of
	all-atom protein structures using generalized belief propagation. J Comput
	Biol. 2008 Sep;15(7):755-66. doi: 10.1089/cmb.2007.0131.
		
	\bibitem{FLY09} M. Fromer, C. Yanover, and M. Linial. Design of multispecific
	protein sequences using probabilistic graphical modeling. Proteins 75;3(2009
	May 15):682-705.
	
	\bibitem{OS1} C. Chen, I. Georgiev, A. C. Anderson, and B. R. Donald. 
	Computational structure-based redesign of enzyme activity. PNAS USA, 106(10):
	3764–3769, 2009.

	\bibitem{OS2} P. Gainza, K. E. Roberts, I. Georgiev, R. H. Lilien, D. A. Keedy,
	C. Chen, F. Reza, A. C. Anderson, D. C. Richardson, J. S. Richardson, and B. R.
	Donald. OSPREY: Protein design with ensembles, flexibility, and provable
	algorithms. Methods in Enzymology, 523:87-107, 2013.
	
	\bibitem{Karmarkar} Komodakis, N.; Tziritas, G., "Approximate Labeling
	via Graph Cuts Based on Linear Programming," Pattern Analysis and Machine
	Intelligence, IEEE Transactions on , vol.29, no.8, pp.1436,1453, Aug. 2007
	doi: 10.1109/TPAMI.2007.1061	
	
	\bibitem{Komodakis} N. Komodakis, G. Tziritas and N. Paragios, "Performance vs
	Computational Efficiency for Optimizing Single and Dynamic MRFs: Setting the 
	State of the Art with Primal Dual Strategies". Computer Vision and Image
	Understanding, 2008 (Special Issue on Discrete Optimization in Computer Vision).
	
	\bibitem{CKNZ} C. Chekuri, S. Khanna, J. Naor, and L. Zosin, “Approximation
	Algorithms for the Metric Labeling Problem via a New Linear Programming
	Formulation,” Proc. 12th Ann. ACM-SIAM Symp. Discrete Algorithms, pp. 109-118,
	2001.
	
	\end{thebibliography}

\end{document}

