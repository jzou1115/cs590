\documentclass[11pt]{article}

\usepackage{amsmath, amssymb}

\title{\textbf{Novel Approaches to the Protein Design Problem}}
\author{Adi Mukund \and Jennifer Zou}
\date{April 20, 2015}

\begin{document}
	
	\maketitle
	
	\section{Introduction}
	The protein design problem can be described as follows. Given a set of 
	backbone coordinates $\mathbf{c} = (\vec{c_1}, \vec{c_2 }, \dots, \vec{c_n})$
	and rotamer library $R$ of length $r$, identify the optimal rotamer
	assignment sequence $\vec{r} = (r_1, r_2,\dots, r_n)$, $r_i \in R$, 
	$1 \leq i \leq n$ according to an energy function $E(\vec{r})$. This assignment
	sequence is known as the global minimum energy conformation (GMEC). 
	This problem has been proven to be NP-hard \cite{PW02}, and algorithms such 
	as DEE \cite{BD97} have been created to allow for combinatorial pruning of 
	the residue search space and make the design problem computationally tractable.
	However, such algorithms cannot guarantee a time complexity less than the 
	worst-case $O(nr^n)$. 
	
	Recent graph based algorithms such as BWM* \cite{DJJG15} use sparse residue 
	interaction graphs in order to more efficiently compute functions over the residue
	space and identify optimal assignments more rapidly. Such graph-based algorithms
	have been able to achieve combinatorial speedups while maintaining provable
	accuracy and returning ensemble of minimum energy conformations. 	
	
	Probabilistic models of protein design assign a probability distribution 
	for rotamers in each position of the protein sequence.  In algorithms such as
	belief propagation, the beliefs or the approximate marginal probabilities
	of each rotamer are computed iteratively.  These approximations are computationally
	less intense, but they are not always provably accurate \cite{KLX08}. Furthermore,
	such algorithms are only guaranteed to converge on tree graphs, which are not
	commonly observed in natural proteins \cite{FLY09}. 

	\section{Graph Cuts and the GMEC}
	
	\subsection{The Graph Labeling Problem}
	
	The goal of this project was to apply graph-based algorithms to the protein
	design problem. We began by attempting to characterize the protein design problem
	as a graph cut problem and identifying relevant algorithms that might allow
	efficient approximations of the GMEC. 
	
	The protein design problem is most accurately represented not as solely a
	graph cut problem, but as a graph labeling problem, where each rotamer is a label.
	The Graph Labeling (GL) problem can be stated as follows: classify a set 
	$\mathcal{V}$ of $n$ objects by assigning to each object a label from a given
	set $\mathcal{L}$ of labels, given a weighted graph
	$\mathcal{G}=(\mathcal{V},\mathcal{E},w)$. For each $p \in \mathcal{V}$ there
	is a label cost $\textbf{c}_{p}(a) \geq 0$ for assigning the label $a=f_p$ to
	$p$, and for every edge $pq$ there is a pairwise cost
	$\textbf{c}_{pq}(a,b) =  w_{pq}d_{pq}(ab)$ where $d_{pq}(ab)$ is the distance
	between (or cost of) label $a$ on vertex $p$ and $b$ on vertex $q$. Thus,
	the cost of a labeling $f$ is as follows:
		
	\begin{equation} \label{k_cost}
			\text{COST}(f) = 
			\sum\limits_{p \in \mathcal{V}} \textbf{c}_{p}(f_p) + 
			\sum\limits_{(p,q) \in \mathcal{E}} w_{pq}d_{pq}(f_p,f_q)
		\end{equation}

	The most promising algorithm for an efficient solution was \cite{Karmarkar}, 
	which provides an approximation algorithm based on graph cuts for the 
	nonmetric labeling problem, which requires a distance function $d(a,b)$ such
	that $d(a,b)=0 \iff a=b$ and $d(a,b) \geq 0$. The algorithm provides a labeling 
	with a cost that is an $f$-approximation to the minimum-cost labeling, where
	$f = \frac{d_{max}}{d_{min}}$, where $d_{max}$ is the maximum distance between
	any two rotamers and $d_{min}$ is the minimum distance. 

	\subsection{Application to the Protein Design Problem}
	
	The internal energy of a protein can easily be modeled by equation \ref{k_cost},
	where the function $\textbf{c}_{p}(f_p)$ is held to represent the internal
	energy of a rotamer $f_p$ at position $p$ and the function 
	$w_{pq}d_{pq}(f_p, f_q)$ is held to represent the pairwise interaction of a
	rotamer $f_p$ at position $p$ and a rotamer $f_q$ at position $q$. 	In addition,
	each residue is modeled by a single node in the vertex set $\mathcal{V}$, and
	interactions between residues are represented by edges between nodes in the edge
	set $\mathcal{E}$. 
	
	In order to maintain the distance constraint 
	$\textbf{c}_{pq}(a,b) =  w_{pq}d_{pq}(ab)$, the label set $L$ was set as the
	Cartesian product $R \times \mathcal{V}$ of the set of all rotamers with the set
	of all positions. Thus, a given label $l \in \mathcal{L}$ represents a specific
	rotamer at a particular position. The pairwise interaction between a label and
	itself can then be set as zero. In order to ensure that a rotamer for position 1
	was not assigned to position 2, the cost of labeling a rotamer to the wrong
	position was set to be prohibitively high. 
	
	\section{Methods}
	
	\subsection{Selection of proteins}
	We selected 3 proteins to investigate. The first, PDB ID FSV1, was the first
	computationally-designed protein structure, and was thus chosen for its relatively
	small size and historical significance. The other two proteins, PDB IDs 1CC8 and
	3F0Q, were chosen for their use as example proteins in the open-source OSPREY
	software suite (as such, they had readily available data against which we could
	compare our graph-based algorithm). 
	
	\subsection{Generation of pairwise interaction matrices}
	OSPREY (\cite{OS1}, \cite{OS2}) was used to generate pairwise interaction matrices
	and provide a baseline against which to compare the results generated by the graph
	-based algorithm.
	
	\subsection{Optimizing the metric labeling problem}
	The FastPD Markov Random Field optimization library 
	(\cite{Karmarkar} and \cite{Komodakis}) was used in order to test the graph
	algorithm. The FastPD algorithm transforms the initial GL problem into the
	minimization of a linear program according to \cite{CKNZ}. 
	
	\begin{equation}
		\begin{split}
		min & \sum_{p \in V} \sum_{a\in L} c_p(a)x_p(a)+ \sum_{(p,q) \in E} w_{pq}
		 \sum_{a,b \in L}d(a,b) x_{pq}(a,b)\\
		s.t. & \sum_a x_p(a) =1 \qquad \forall p \in V\\
		& \sum_a x_{pq}(a,b)=x_q(b) \qquad \forall b \in L, (p,q) \in E\\
		& \sum_b x_{pq}(a,b)=x_p(a) \qquad \forall a \in L, (p,q) \in E
		\end{split}
	\label{linear}
	\end{equation}
	
	Equation \ref{linear} shows a formulation of an integer linear program, where
	$x_p(a)$ is a binary variable that indicates whether vertex p is assigned to
	label a, and $x_{pq}(a,b)$ indicates that vertices p and q are assigned the
	labels a and b.  The first constraint requires each vertex to be assigned to
	one label, and the other constraints maintains consistency between the labeled
	vertices and edges.
	
	Optimizing the cost of the metric labeling problem is NP-hard, and many
	approximation algorithms ($\alpha$-expansion, $\alpha$-$\beta$-swap) either
	assume metric distances that satisfy the triangle inequality or provide no
	guarantees about optimality.  Since pairwise interaction energies between
	protein residues are inherently nonmetric distances, these algorithms are not
	adequate for the protein design problem.  
	
	FastPD only requires a nonmetric distance function that satisfies $d(a,b)$ such
	that $d(a,b)=0 \iff a=b$ and $d(a,b) \geq 0$.  Since the pairwise interaction
	energy between a residue and itself is zero, and all energies are greater than
	zero, this algorithm can be applied to the protein design problem.  
	
	Relaxation of the integer linear program constraints transforms the NP-hard
	optimization problem into one that can be solvable in polynomial time.  The 
	FastPD algorithm modifies the constraints to 
	$x_p(\cdot) \ge 0$ and $x_{pq}(\cdot, \cdot) \ge 0$.  

	\subsection{The Primal-Dual Schema}	
	
	The FastPD algorithm utilizes the primal-dual schema in order to rapidly obtain
	an $f$-approximation to the optimal solution. Given a primal program
	\begin{gather*}
	\text{min }\textbf{c}^T\textbf{x} \\
	\text{s.t. }\textbf{Ax} = \textbf{b}, \textbf{x} \geq 0
	\end{gather*}
	the dual program can be formulated as follows:
	\begin{gather*}
	\text{max } \textbf{b}^T\textbf{y} \\
	\text{s.t. }\textbf{A}^T\textbf{y} \leq \textbf{c} 
	\end{gather*}		

	The primal-dual principle states that if $\textbf{x}$ and $\textbf{y}$ are
	solutions satisfying $\textbf{c}^T\textbf{x} \leq f \cdot \textbf{b}^T\textbf{y}$,
	then $\textbf{x}$ is an $f$-approximation to the optimal solution $\textbf{x}*$.
	FastPD starts with initial guesses for $\textbf{x}$ and $\textbf{y}$ and 
	repeatedly improves them by making every variable in the program a node in the
	graph and using a max-flow/min-cut algorithm to alter the assigned values to
	each variable. The algorithm converges to an approximation factor
	$f_{app} = 2 \frac{d_{max}}{d_{min}}$ where $d_{max}$ is the largest distance
	between two labels and $d_{min}$ is the smallest distance between two labels. 
	
	Thus, the FastPD algorithm returns a conformation whose energy is an $f$-
	approximation to the GMEC. This project aimed to test whether or not this would
	be sufficiently accurate so as to produce useful results. 

	\section{Results}
	
	\begin{enumerate}
		\item DHFR
		\item 1CC8
		\item 1FSV\\
		Position 10 got position 27's glutamine\\
		Position 27 got position 27's histidine
	\end{enumerate}

	\section{Limitations}
	
	\subsection{Residue Labeling}
	There were numerous problems with applying the FastPD algorithm to the protein
	design problem. Most notably, for the 1FSV and 1CC8 proteins both resulted in
	labels applied to the wrong residue. For 1FSV, allowing mutations at residue
	10 and 27 resulted in the algorithm attempting to place the glutamine reserved
	for residue 27 at residue 10. While we did attempt to prevent this by increasing
	the cost of applying a label to the wrong residue to the maximum value possible,
	this strategy was evidently unsuccessful. 
	
	The most likely cause of this is the $f$-approximation produced by the FastPD
	algorithm, where $f = 2 \frac{d_{max}}{d_{min}}$, where the distance functions
	represent pairwise interaction energies. Because the maximum pairwise interaction
	energy would be immensely high (representing steric clash), and because the
	minimum pairwise interaction energy would be extremely low (representing the 
	stabilizing influence of, for example, a hydrogen bond), the value of $f$ will
	be far too high to return a useful approximation to the GMEC. 
	
	\subsection{Further Implications}
	The results from this indicate that, on the whole, applying graph-based algorithms
	designed for image segmentation problems may not be a particularly fruitful
	endeavor. There are three reasons for this. 
	
	\subsubsection{$f$-Approximation}
	Typical graph-based approximation algorithms return an $f$-approximation to
	the optimal solution \cite{CKNZ}. The viability of this approach relies
	on an assumption that ``good is good enough'' with regards to the returned
	solution.  However, due to the multitude of local minima observed in the
	conformational landscape of any peptide \cite{DCM0}, merely being \textit{close}
	to the optimal configuration is not sufficient. Thus, while comparatively
	efficient $f$-approximations may be useful in image segmentation problems 
	when slightly suboptimal solutions may not be of extreme consequence, even
	slight gaps between the returned solution and optimal configuration are of
	significant concern. 
	
	\subsubsection{Multiple Labels}
	In addition, while the algorithm used in this project featured an approximation
	that scaled with the ratio of pairwise interaction distances, more commonly 
	used algorithms feature approximations that scale with the number of possible
	labels on the graph \cite{BVZ1}. In order to maintain distance constraints that
	permit the application of such algorithms, the number of labels must be 
	multiplied by the number of residues, as was done here. However, this in turn
	reduces the accuracy of the approximation, making graph-based algorithms less
	useful for the protein design problem. 	
	
	\section{Conclusion}
	This project attempted to apply graph-based algorithms to the protein design
	problem to see if such approaches held any promise for future improvements
	to current algorithms. We applied the FastPD Markov Random Field optimization
	library to three proteins, with varying numbers of mutable residues, and found
	that the approximations returned were poor enough so as to be functionally
	meaningless. These results seem to indicate that similar graph-based approaches
	that provide factor-approximations to the optimal solution are poor choices for
	further investigation with regards to the protein design problem. It remains
	to be seen whether efficient $\epsilon$-approximation algorithms exist, and,
	if so, whether they might be feasible options instead. 
	
	\begin{thebibliography}{99}
	
	\bibitem {PW02} Pierce, Niles A. and Winfree, Erik. Protein Design is NP-hard 
	Protein Eng. (2002) 15 (10): 779-782 doi:10.1093/protein/15.10.779
	
	\bibitem{BD97} Dahiyat, B. I. De Novo Protein Design: Fully Automated Sequence
	Selection. Science 278, 82-87 (1997)	

	\bibitem{DJJG15} Jou, J.D. Jain, S. Georgiev, I. Donald, B.R. BWM*: A Novel,
	Provable, Ensemble-based Dynamic Programming Algorithm for Sparse 
	Approximations of Computational Protein Design. RECOMB (2015) Warsaw, Poland.
	April 12, 2015 (In Press)

	\bibitem{KLX08} Kamisetty H1, Xing EP, Langmead CJ. Free energy estimates of
	all-atom protein structures using generalized belief propagation. J Comput
	Biol. 2008 Sep;15(7):755-66. doi: 10.1089/cmb.2007.0131.
		
	\bibitem{FLY09} M. Fromer, C. Yanover, and M. Linial. Design of multispecific
	protein sequences using probabilistic graphical modeling. Proteins 75;3(2009
	May 15):682-705.
	
	\bibitem{OS1} C. Chen, I. Georgiev, A. C. Anderson, and B. R. Donald. 
	Computational structure-based redesign of enzyme activity. PNAS USA, 106(10):
	3764–3769, 2009.

	\bibitem{OS2} P. Gainza, K. E. Roberts, I. Georgiev, R. H. Lilien, D. A. Keedy,
	C. Chen, F. Reza, A. C. Anderson, D. C. Richardson, J. S. Richardson, and B. R.
	Donald. OSPREY: Protein design with ensembles, flexibility, and provable
	algorithms. Methods in Enzymology, 523:87-107, 2013.
	
	\bibitem{Karmarkar} Komodakis, N.; Tziritas, G., "Approximate Labeling
	via Graph Cuts Based on Linear Programming," Pattern Analysis and Machine
	Intelligence, IEEE Transactions on , vol.29, no.8, pp.1436,1453, Aug. 2007
	doi: 10.1109/TPAMI.2007.1061	
	
	\bibitem{Komodakis} N. Komodakis, G. Tziritas and N. Paragios, "Performance vs
	Computational Efficiency for Optimizing Single and Dynamic MRFs: Setting the 
	State of the Art with Primal Dual Strategies". Computer Vision and Image
	Understanding, 2008 (Special Issue on Discrete Optimization in Computer Vision).
	
	\bibitem{CKNZ} C. Chekuri, S. Khanna, J. Naor, and L. Zosin, “Approximation
	Algorithms for the Metric Labeling Problem via a New Linear Programming
	Formulation,” Proc. 12th Ann. ACM-SIAM Symp. Discrete Algorithms, pp. 109-118,
	2001.
	
	\bibitem{DCM0} Dobson CM (2000-12-15). "The nature and significance of protein
	folding". In RH Pain. Mechanisms of Protein Folding (2nd ed.). Oxford, UK: Oxford
	University Press. ISBN 0-19-963788-1.	
	
	\bibitem{BVZ1} Y. Boykov, O. Veksler, R. Zabih. Fast approximate energy
	minimization via graph cuts. IEE Transactions on Pattern Analysis and Machine
	Intelligence, vol. 23, no. 11, pp. 1222-1239. Nov 2001. 
		
	\end{thebibliography}

\end{document}

