\documentclass[11pt]{article}

\usepackage{amsmath, amssymb}

\title{\textbf{Graph Cuts and Protein Design}}
\author{Adi Mukund \and Jennifer Zou}
\date{}

\begin{document}

	\maketitle
	
	\section{The Protein Design Problem}
	
	The protein design problem can be described as follows. Given a set of 
	backbone coordinates $\mathbf{c} = (\vec{c_1}, \vec{c_2 }, \dots, \vec{c_n})$
	and rotamer library $R$ of length $r$, identify the optimal rotamer
	assignment sequence $\vec{r} = (r_1, r_2,\dots, r_n)$, $r_i \in R$, 
	$1 \leq i \leq n$ according to an energy function $E(\vec{r})$. This assignment
	sequence is known as the global minimum energy conformation (GMEC). 
	
	\section{Graph Labeling}
	
	The graph-cut formulation of the protein design problem represents the set
	of all residues as a sparse interaction graph, a modification of the residue
	interaction graph $G = (V, \mathcal{E})$ in which each vertex represents a
	protein residue and each edge represents an interaction between two such
	residues. The sparse graph is created by omitting a set of interactions
	$\mathcal{E}^*$ to create the graph $G^* = (V, \mathcal{E}-\mathcal{E}^*)$.
	
	The label graph $G_l$ is generated by adding a set of vertices $V_r$ 
	representing all of the rotamers in the rotamer library R to the vertex
	set $V^*$ of $G_*$ and connecting each vertex in $V^*$ to each vertex in
	$V_r$ to generate the vertex set $V_l$ and corresponding edge set
	$\mathcal{E}_l$. The set of all edges connecting vertices in $V_r$ and $V^*$
	is $\mathcal{E}_r$. Edges from $\mathcal{E}_l$ are then pruned until every 
	vertex in $V^*$ is connected to exactly 1 vertex in $V_r$ (that is, until 
	every member of $V^*$ is represented in $\mathcal{E}_r$ exactly once) in
	order to generate a labeling for the $G$ where each vertex in $V^*$ is 
	connected to a vertex in $V_r$ representing its optimal rotameric assignment. 
	
	The energy of the graph can then be computed by finding
	\begin{equation} \label{e_g}
	    E_G = 
	    \sum\limits_{r \in V_r} \sum\limits_{v \in V^*} \phi(r,v) +
	    \sum\limits_{v \in V^*} \sum\limits_{\substack{w \in V^* \\ w \neq v}} \psi(v,w)
	\end{equation}
	where $\phi(r,v)$ represents the cost of assigning label $r$ to vertex $v$
	(intuitively, the internal energy of residue $v$ when assigned rotamer $r$)
	and $\psi(v,w)$ represents the pairwise interaction between vertices $v$ and
	$w$ given an initial labeling.	
	
	Thus, finding the GMEC is equivalent to minimizing equation \ref{e_g}
	over the set $\mathcal{L}$ of all possible labelings. 
	
	\section{ILP Transformation}
	
	\section{Approximation Bounds}

\end{document}