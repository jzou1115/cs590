\documentclass[11pt]{article}

\usepackage{amsmath, amssymb}

\title{\textbf{Graph Cuts and Protein Design}}
\author{Adi Mukund \and Jennifer Zou}
\date{}

\begin{document}

	\maketitle
	
	\section{The Protein Design Problem}
	
	The protein design problem can be described as follows. Given a set of 
	backbone coordinates $\mathbf{c} = (\vec{c_1}, \vec{c_2 }, \dots, \vec{c_n})$
	and rotamer library $R$ of length $r$, identify the optimal rotamer
	assignment sequence $\vec{r} = (r_1, r_2,\dots, r_n)$, $r_i \in R$, 
	$1 \leq i \leq n$ according to an energy function $E(\vec{r})$. This assignment
	sequence is known as the global minimum energy conformation (GMEC). 
	
	\section{Graph Labeling}
	
	The graph-label formulation of the protein design problem represents the set
	of all residues as a sparse interaction graph, a modification of the residue
	interaction graph $G = (V, \mathcal{E})$ in which each vertex represents a
	protein residue and each edge represents an interaction between two such
	residues. The sparse graph is created by omitting a set of interactions
	$\mathcal{E}^*$ to create the graph $G^* = (V, \mathcal{E}-\mathcal{E}^*)$.
	
	The label graph $G_l$ is generated by adding a set of vertices $V_r$ 
	representing all of the rotamers in the rotamer library R to the vertex
	set $V^*$ of $G_*$ and connecting each vertex in $V^*$ to each vertex in
	$V_r$ to generate the vertex set $V_l$ and corresponding edge set
	$\mathcal{E}_l$. The set of all edges connecting vertices in $V_r$ and $V^*$
	is $\mathcal{E}_r$. Edges from $\mathcal{E}_l$ are then pruned until every 
	vertex in $V^*$ is connected to exactly 1 vertex in $V_r$ (that is, until 
	every member of $V^*$ is represented in $\mathcal{E}_r$ exactly once) in
	order to generate a labeling for the $G$ where each vertex in $V^*$ is 
	connected to a vertex in $V_r$ representing its optimal rotameric assignment.  
	
	Given a function $\phi(v, r_v)$ that returns the cost of assigning label $r_v$
	to vertex $v$ and a function $\psi(v, r_v, w, r_w)$ that returns the pairwise
	interaction between residues $v$ and $w$ when assigned labels $r_v$ and $r_w$
	respectively, the energy of the graph (and associated rotamer assignment 
	sequence) can be computed as follows:
	
	The energy of the graph can then be computed by finding the sum over the rotamers 
	$v,w\in V^*$ where $w > v$ and their associated labels $r_v$ and $r_w$ as
	follows:
	
	\begin{equation} \label{e_g}
	    E_G = 
	    \sum\limits_{v} \phi(v, r_v) + 
	    \sum\limits_{v} \sum\limits_{w > v} \psi(v,r_v, w, r_w)
	   \end{equation}
	
	Thus, finding the GMEC is equivalent to minimizing equation \ref{e_g}
	over the set of all possible labelings of $G_l$. 
	
	\section{Approximate Labeling based on Linear Programming}
	
	\textit{Note: This section is based on the paper ``Approximate Labeling via
		Graph Cuts Based on Linear Programming'' by Komodakis and Tziritas.}
	
	\subsection{Metric Labeling}
	
	The Metric Labeling (ML) problem can be stated as follows: classify a set 
	$\mathcal{V}$ of $n$ objects by assigning to each object a label from a given
	set $\mathcal{L}$ of labels, given a weighted graph
	$\mathcal{G}=(\mathcal{V},\mathcal{E},w)$. For each $p \in \mathcal{V}$ there
	is a label cost $\textbf{c}_{p}(a) \geq 0$ for assigning the label $a=f_p$ to
	$p$, and for every edge $pq$ there is a pairwise cost
	$\textbf{c}_{pq}(a,b) =  w_{pq}d_{pq}(ab)$ where $d_{pq}(ab)$ is the distance
	between (or cost of) label $a$ on vertex $p$ and $b$ on vertex $q$. Thus,
	the cost of a labeling $f$ is as follows:
	
	\begin{equation} \label{k_cost}
	    \text{COST}(f) = 
	    \sum\limits_{p \in \mathcal{V}} \textbf{c}_{p}(f_p) + 
	    \sum\limits_{(p,q) \in \mathcal{E}} w_{pq}d_{pq}(f_p,f_q)
	   \end{equation}

	\subsection{The Primal-Dual Schema}
	
	Given the primal program $\{\min{\textbf{c}^{T}\textbf{x}} \mid \textbf{Ax}=
	\textbf{b}, \textbf{x}\geq 0\}$ it is possible to formulate the dual program
	$\{\max{\textbf{b}^{T}\textbf{y}} \mid \textbf{A}^T\textbf{y} \leq \textbf{c}\}$.
	The primal-dual principle states that if 
	$\textbf{c}^T\textbf{x} \leq f \cdot \textbf{b}^T\textbf{y}$, then \textbf{x}
	is an $f$ -approximation to the optimal integral solution 
	$\textbf{x}^{\text{*}}$, that is, 
	$\textbf{c}^T\textbf{x}^\text{*} \leq \textbf{c}^T\textbf{x} \leq 
        	f \cdot \textbf{c}^T\textbf{x}^\text{*}$. 
        	
    The Primal-Dual Schema is to keep generating pairs of primal and dual solutions
   	$\{ ( \textbf{x}^k, \textbf{y}^k ) \}_{k=1}^{t} $ until the pair
   	$(\textbf{x}^t, \textbf{y}^t)$ are both feasible and satisfy the relaxed
   	primal complementary slackness conditions. 
    
    Some shenanigans then occur, and I have no idea how it all works, but it seems
    that a (roughly) 1.4-approximation can be found. It might be useful to 
    implement this and see if we get better results empirically. 
    	
	\section{A* all the things}
	
	Just A* over the cut space - maintain a sorted list of partial cuts, 
	expanding until we get the best cut. This also guarantees ensemble 
	enumeration! Though really this is pretty pointless.

\end{document}